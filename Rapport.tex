\documentclass{article}

\usepackage{amsmath}
\usepackage{amsthm}
\usepackage{amssymb}


\title{Étude du théorème de Hahn-Banach dans les espaces vectoriels topologiques}
\author{LAMLIH Houssam}
\date{}

\begin{document}

\maketitle

\section{Introduction}

Dans ce travail, nous étudions le théorème de Hahn-Banach dans sa forme 
géométrique, en nous intéressant aux espaces vectoriels topologiques dont 
la topologie est issue d'une métrique. Plus particulièrement, nous examinons
le cas des espaces vectoriels normés traité dans \cite{brezis1983analyse} 
et le cas plus général des espaces vectoriels topologiques abordé dans 
\cite{schwartz1970topologie}. 

\section{Théorème de Hahn-Banach dans les espaces vectoriels normés}
\begin{proof}[Preuve du théorème de Hahn-Banach]
    On note $E$ l'ensemble des couples $(V, u)$ tels que $V$ soit un sous-espace vectoriel de $E$ contenant $F$, et $u$ une forme linéaire sur $V$ qui coïncide avec $L$ sur $F$ et vérifie $u(x) \leq p(x)$ pour tout $x \in V$. On munit $E$ de la relation d'ordre $\leq$ définie par $(V_1, u_1) \leq (V_2, u_2)$ si $V_1 \subset V_2$ et $u_2 = u_1$ sur $V$.
    
    L'ensemble $(E, \leq)$ est ordonné par construction. De plus, il est non vide car contient $L$ et inductif car si $\{(V_i, u_i), i \in I\}$ est une partie totalement ordonnée de $E$, alors $\bigcup_{i \in I} V_i$ est un sous-espace vectoriel de $E$ contenant $F$, et l'endomorphisme $u$ défini sur $\bigcup_{i \in I} V_i$ par $u(x) = u_i(x)$ si $x \in V_i$ est bien un majorant de la partie $\{(V_i, u_i), i \in I\}$.
    
    D'après le lemme de Zorn, $E$ admet donc un élément maximal $(V, L_e)$. Supposons par l'absurde que $V \neq E$. Alors $E \setminus V$ contient au moins un élément non nul $x$. Pour un réel donné $a$, on définit alors une forme linéaire $L_{ea}$ sur $V \oplus \mathbb{R}$ par $L_{ea}(y) = L_e(y)$ si $y \in V$, et $L_{ea}(x) = a$.
    
    Grâce aux propriétés de $L_e$ et de $p$, on peut ajuster $a$ de telle sorte que $\forall y \in V, \forall \lambda \in \mathbb{R}, L_{ea}(y + \lambda x) = L_e(y) + \lambda a \leq p(y + \lambda x)$. En effet, cette inégalité est clairement vérifiée pour tout $y \in V$ si $\lambda = 0$. De plus, l'inégalité restreinte aux $\lambda > 0$ est équivalente à $\forall y \in V, L_e(y) + a \leq p(y + x)$, alors que pour $\lambda < 0$, elle est équivalente à $\forall z \in V, L_e(z) - a \leq p(z - x)$. Finalement, cette inégalité est donc vérifiée si et seulement si $a$ est choisi de telle sorte que
    \[
    \sup_{z \in V} (L_e(z) - p(z - x)) \leq a \leq \inf_{y \in V} (p(y + x) - L_e(y)).
    \]
    
    Comme pour tout $(y, z) \in V^2$, on a $L_e(y) + L_e(z) \leq p(y + z) \leq p(y + x) + p(z - x)$, un tel choix de $a$ est possible.
    
    On a donc $(V, L_e) \leq (V \oplus \mathbb{R}, L_{ea})$. Comme $x \notin V$, cela contredit la maximalité de $(V, L_e)$.
    
    \end{proof}
    
    
\section{Espaces vectoriels topologiques}

Nous introduisons tout d'abord la notion d'espace vectoriel topologique 
(EV-Top) et présentons des exemples concrets, notamment les espaces 
vectoriels normés. Nous étudions ensuite les propriétés des voisinages 
de zéro et des voisinages d'un point dans un EV-Top. Nous discutons 
également les espaces vectoriels topologiques séparés et démontrons que 
l'adhérence d'un sous-espace vectoriel est un sous-espace vectoriel. 
Enfin, nous abordons la notion de quotient $E/F$ dans un EV-Top.

\section{Théorème de Hahn-Banach dans les espaces vectoriels topologiques}

Dans cette dernière partie, nous nous concentrons sur le théorème de 
Hahn-Banach dans le cas des espaces vectoriels topologiques. Nous 
éclaircissons le théorème (T.2, XIX,6; 1) mentionné dans 
\cite{schwartz1970topologie}, ce qui nécessite l'étude de résultats 
supplémentaires sur les espaces vectoriels topologiques. Nous nous 
appuyons sur différentes références, y compris \cite{nier2002introduction} 
et \cite{dixmier1981topologie}, pour obtenir une vision complète du sujet.

\section{Conclusion}

En conclusion, nous avons étudié le théorème de Hahn-Banach dans sa 
forme géométrique pour les espaces vectoriels topologiques. Nous avons 
examiné le cas des espaces vectoriels normés, en nous référant à 
\cite{brezis1983analyse}, et le cas plus général des EV-Top, en 
utilisant les références \cite{schwartz1970topologie}, 
\cite{nier2002introduction}, et \cite{dixmier1981topologie} pour 
approfondir nos connaissances.

\bibliographystyle{plain}
\bibliography{references}

\end{document}
