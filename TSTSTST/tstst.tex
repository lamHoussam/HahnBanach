\documentclass{article}
\usepackage{amsmath}
\usepackage{amsfonts}


\begin{document}


% Soit $E$ un espace vectoriel topologique sur un corps $\mathbb{K}$ (où $\mathbb{K}$ peut être les réels $\mathbb{R}$ ou les complexes $\mathbb{C}$) et $F$ un sous-espace vectoriel de $E$. Soit $p : E \rightarrow \mathbb{R}$ une semi-norme sur $E$ et $\ell_0 : F \rightarrow \mathbb{K}$ une forme linéaire définie sur $F$. Si $\ell_0$ est bornée par rapport à $p$, c'est-à-dire qu'il existe une constante réelle $M \geq 0$ telle que $|\ell_0(x)| \leq M \cdot p(x)$ pour tout $x \in F$, alors il existe une extension linéaire continue $\ell : E \rightarrow \mathbb{K}$ de $\ell_0$ telle que $|\ell(x)| \leq M \cdot p(x)$ pour tout $x \in E$.

\end{document}